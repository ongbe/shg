% bdstest.tex: BDS test for independence
\documentclass[a4paper,11pt]{article}
\usepackage[T1]{fontenc}
\usepackage{amsmath}
%\renewcommand{\Pr}{\mathit{P}}

\title{BDS test for independence}
\author{Stanis{\l}aw Galus}
\date{27 October 2013}

\begin{document}
\maketitle

\section{Name}

bdstest --- BDS test for independence

\section{Synopsis}

{\small \begin{verbatim}
#include "shg/bdstest.h"
using namespace SHG;
class BDS_test {
public:
     struct Result {
          double stat;
          double pval;
     };
     BDS_test(const std::vector<double>& u,
              int maxm,
              const std::vector<double>& eps);
     BDS_test(const std::vector<double>& u);
     inline int maxm() const;
     inline const std::vector<double>& eps() const;
     inline const std::vector<std::vector<Result>>& res() const;
private:
     /* ... */
};

std::ostream& operator<<(std::ostream& stream, const BDS_test& b);
\end{verbatim}}

\section{Description}

The class performs the BDS test for independence~\cite
{brock-dechert-scheinkman-lebaron-1996}.

Let $(u_t)_{t = 1}^n$ be a time series. The BDS test statistic for an
embedding dimension $m \geq 1$ and a threshold $\epsilon > 0$ is
defined as
\begin{equation} \label{eq:statistic}
  W(m, \epsilon) = \sqrt{n} \, \frac{C_{m, n}(\epsilon) - [C_{1,
        n}(\epsilon)]^m} {V_{m, n}(\epsilon)},
\end{equation}
where\footnote{$\chi_{\epsilon}$ is the characteristic function of the
  interval $[0, \epsilon)$.}
\begin{equation} \label{eq:Cmn}
  C_{m, n}(\epsilon) = \frac{2}{(n - m + 1)(n - m)} \sum_{1 \leq s < t
    \leq n - m + 1} \chi_{\epsilon} \left( \max_{0 \leq i < m} |u_{s +
    i} - u_{t + i}| \right)
\end{equation}
is the correlation integral\footnote{Cf.~\cite [p.~120]
  {baker-gollub-1998}.} and
\begin{equation} \label{eq:bds2.11}
  \begin{split}
  \frac{1}{4}V_{m, n}^2(\epsilon) &= m(m - 2) C^{2m - 2} (K - C^2) +
  K^m - C^{2m} + \\
  & \quad + 2 \sum_{j = 1}^{m - 1} \left[ C^{2j} (K^{m - j} - C^{2m -
      2j}) - mC^{2m -2} (K - C^2) \right],
  \end{split}
\end{equation}
where
\begin{align} \label{eq:bds2.12}
  C &= C_n(\epsilon) = \frac{1}{n^2} \sum_{s = 1}^n \sum_{t = 1}^n
  \chi_{\epsilon}(|u_s - u_t|), \\
  \label{eq:bds2.13}
  K &= K_n(\epsilon) = \frac{1}{n^3} \sum_{r = 1}^n \sum_{s = 1}^n
  \sum_{t = 1}^n \chi_{\epsilon}(|u_r - u_s|) \chi_{\epsilon}(|u_s -
  u_t|).
\end{align}
If $(u_t)_{t = 1}^n$ is a series of independent identically
distributed random variables, then for $m \geq 2$ the
statistic~(\ref{eq:statistic}) converges in distribution to the
standard normal distribution.

The first constructor requires the time series $(u_t)_{t = 1}^{n}$,
arranged in a vector with index running from 0 to $n - 1$, the maximum
embedding dimension $\mathit{maxm}$ and the vector of thresholds
$\mathit{eps}$. The second constructor requires only the time series
and arbitrarily sets $\mathit{maxm} = 8$ and \[ \mathit{eps}
= \begin{bmatrix} 0.5s & 0.75s & s & 1.25s & 1.5s & 1.75s &
  2s \end{bmatrix}, \] where
\begin{equation} \label{eq:mean_var}
  s^2 = \frac{1}{n} \sum_{t = 1}^n (u_t - \bar{u})^2, \quad
  \bar{u} = \frac{1}{n} \sum_{t = 1}^n u_t.
\end{equation}

After successful construction, the function \verb|res()| returns an
array of structures of type \verb|BDS_test::Result|, whose member
\verb|res()[m][i].stat| reports the value of the statistic $W(m,
\epsilon)$ defined by~(\ref{eq:statistic}) for the embedding dimension
$2 \leq m \leq \mathit{maxm}$ and the $i$-th threshold in
$\mathit{eps}$. The member \verb|res()[m][i].pval| reports the
probability
\[ \left\{ \begin{array}{ll}
  \Phi(W(m, \epsilon)) & \mbox{if $W(m, \epsilon) < 0$,} \\
  1 - \Phi(W(m, \epsilon)) & \mbox{if $W(m, \epsilon) \geq 0$,}
\end{array} \right. \]
where $\Phi$ is the cumulative distribution function of the standard
normal distribution. The values of \verb|res()[m][i]| are undefined
for $m = 0, 1$.

The functions \verb|maxm()| and \verb|eps()| return $\mathit{maxm}$
and the vector $\mathit{eps}$ used during construction, respectively.

The operator \verb|<<| outputs the four-column plain table of results
with $\mathit{\epsilon}$, $m$, $W(m, \epsilon)$ and the p-value on
each row.

\section{Implementation}

In the implementation,~(\ref{eq:bds2.11}) is simplified to
\begin{equation} \label{eq:simpleV}
  \frac{1}{4}V_{m, n}^2(\epsilon) = K^m + (m - 1)^2 C^{2m} - m^2 K
  C^{2m -2} + 2 \sum_{j = 1}^{m - 1} C^{2j} K^{m - j}
\end{equation}
and~(\ref{eq:bds2.13}) is simplified to
\begin{equation} \label{eq:bds2.13s}
  K = \frac{1}{n^3} \sum_{s = 1}^n \left[ \sum_{t = 1}^n
    \chi_{\epsilon}(|u_s - u_t|) \right]^2.
\end{equation}

\section{Errors}

The constructors can throw \verb|std::invalid_argument| if $n < 1$ or
$n$ is too big or $\mathit{maxm} < 2$ or $\mathit{maxm} \geq n$. They
can also throw \verb|std::range_error| if due to rounding errors
$V_{m, n}^2(\epsilon) / n$, required to
calculate~(\ref{eq:statistic}), is negative or too small or the
variance in~(\ref{eq:mean_var}) is negative or too small.

\section{Example}

The following program:

{\footnotesize \begin{verbatim}
#include <iostream>
#include <iomanip>
#include "shg/bdstest.h"

/** Borosh-Niederreiter random number generator. See Donald E. Knuth,
    Sztuka programowania. Tom 2. Algorytmy seminumeryczne, WNT,
    Warszawa 2002, p. 113. */
double bn() {
     static unsigned long int x = 1ul;
     x = (1812433253ul * x) & 0xfffffffful;
     return x / 4294967296.0;
}

using namespace std;
using SHG::BDS_test;

int main() {
     vector<double> u(1000);
     cout << fixed << setprecision(5);

     for (vector<double>::size_type i = 0; i < u.size(); i++)
          u[i] = bn();
     /** eps is the standard deviation of U(0, 1). */
     cout << BDS_test(u, 8, {sqrt(1.0 / 12.0)}) << '\n';

     for (vector<double>::size_type i = 0; i < u.size(); i++)
          u[i] = i % 2 ? 2.0 * bn() : bn();
     /** eps is the standard deviation of the mixture of U(0, 1) and
         U(0, 2) with mixing weights 0.5. */
     cout << BDS_test(u, 8, {sqrt(13.0 / 48.0)});
}
\end{verbatim}}
\noindent produces:
{\footnotesize \begin{verbatim}
0.28868 2 0.27392 0.39207
0.28868 3 0.26732 0.39461
0.28868 4 -0.33474 0.36891
0.28868 5 -0.97089 0.16580
0.28868 6 -1.83736 0.03308
0.28868 7 -2.35252 0.00932
0.28868 8 -2.16494 0.01520

0.52042 2 -3.96242 0.00004
0.52042 3 0.39043 0.34811
0.52042 4 -0.07102 0.47169
0.52042 5 1.30413 0.09609
0.52042 6 1.26937 0.10215
0.52042 7 2.17663 0.01475
0.52042 8 2.04631 0.02036
\end{verbatim}}

\bibliographystyle{plain}
\bibliography{sgalus}
\end{document}
